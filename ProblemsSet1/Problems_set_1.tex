\documentclass{scrartcl}
\usepackage[utf8]{inputenc}

\title{Algorithms on Strings}
\subtitle{Problems Set 1}
\author{Aleksander Czeszejko-Sochacki}
\date{October 2018}
\usepackage{amsmath}
\usepackage{amsthm}
\usepackage{amssymb}
\newtheorem{lema}{Lemma}

\begin{document}

\maketitle

\section{}
  \begin{proof}
    \begin{equation}\label{eq:1}
      \begin{split}
        |w| - p \textit{ is a border } &\equiv w[1, |w| - p] = w[|w| - (|w| - p), |w|] \\
        &\equiv w[1, |w| - p] = w[p, |w|] \\
        &\equiv w[i] = w[i + p] \textit{ for } i \in \{1, 2, \dots, |w| - p\} \\
        &\equiv \textit{ p is a period of w}
      \end{split}
    \end{equation}
  \end{proof}
\section{}

  \subsection{}
    \begin{proof}
      \begin{equation}
        \begin{split}
          \textit{p is a period of w } &\equiv 
          \textit{w is a subword of some } x^k \textit{ with } |x| = p \textit{ and } k > 0 \\ 
          &\equiv w = ux^jv \textit{ where } j \leq k, u \sqsubset x \textit{ and } v \sqsupset x
        \end{split}
      \end{equation}

      Since u and v - suffix and prefix of x, respectively, we can proceed as follows:

      \begin{equation}\label{eq:2}
        w = ux^jv = u(zu)^jv = (uz)^juv
      \end{equation}
    
      Hence, the period condition holds for i $\in \{1, 2, \dots, |w| - p - |v|\}$. On the other hand:

      \begin{equation}\label{eq:3}
        w = ux^jv = ux^{j - 1}vyv
      \end{equation}

      So the period condition holds for i $\in \{|w| - p - |v|, |w| - p - |v| + 1, \dots, |w| - p\}$.
      From \ref{eq:2} and \ref{eq:3} the proof is done.
    \end{proof}
    
  \subsection{}
    \begin{proof}
      Assume $|w| = kp + l$. From the period condition we have:
      
      \begin{equation}\label{eq:5}
        w[1, p] = w[p + 1, 2p] = \dots = w[(k - 1]p + 1, kp]
      \end{equation}
      
      and
      
      \begin{equation}\label{eq:6}
        w[kp + 1, kp + l] = w[(k - 1)p + 1, (k - 1)p + l]
      \end{equation}
      
      Basing on \ref{eq:5}, we can write $w = y^ku$, such that $|y| = p$ and $|u| = l$. On the other hand, by \ref{eq:6}, we have $w = xuvu$, where $|u| = l$, $|v| = p - l$, $|x| = |w| - p - l$. Combining these two conclusions, 
$y = uv$, so $w = (uv)^ku$.
    \end{proof}
    
  \subsection{}
    \begin{proof}
      \begin{equation}
        \begin{split}
          \textit{p is a period of w} &\equiv |w| - p \textit{ is a border of w (\ref{eq:1})} \\ &\equiv  \exists_\textit{x, y, z}(|y| = |w| - p \land xy = yz = w) \\ &\equiv \exists_\textit{x, y, z}(|x| = |y| = p \land xy = yz = 
w)
        \end{split}
      \end{equation}
    \end{proof}
    
    \begin{lema}[uv periods]\label{lemma:10}
       For any unempty $u, v$ and some $k \in \mathbb{N}$, if $uv = vu$, then $|u|,|v|$ are periods of $(uv)^k$.
    \end{lema}
    \begin{proof}
      \begin{equation}
        \begin{split}
          |v| \textit{ is a period of } (uv)^k &\equiv |uv|^k - |v| \textit{ is a border of } (uv)^k 
          (\ref{eq:1}) \\ &\equiv (uv)^{k-1}u \textit{ is both prefix and suffix of } (uv)^k
        \end{split}
      \end{equation}
      
      Indeed, $(uv)^{k-1}u \sqsubset (uv)^k$ and $(uv)^{k-1}u \sqsupset (vu)^k = (uv)^k$ For $|u|$ the proof is simmilar.
    \end{proof}
    
  \section{}
    \begin{proof}
      The thesis is equvalent to the following implication:
      \[\textit{p, q are periods of } w \implies \textit{q mod p, p are periods of w}\]
      
      Let $w[1, p] = P$, $p < q$ ($p = q$ does not make any sense). We can write:
      \[w = w[1, q]w[q + 1, |w|] = \underbrace{P^ku}_{w[1, q]}\underbrace{vP^ky}_{w[q + 1, |w|]}\]
      where $uv = P$, $k,l \in \mathbb{N}$.
      
      \begin{equation}
        \begin{split}
          q \textit{ is a period of } w &\implies |w| - q \textit{ is a border of } w \\ &\implies
          w[q + 1, |w|] \sqsubset w \\ &\equiv vP^ky \sqsubset w
        \end{split}
      \end{equation}
      
      As we know from the problem conditions, $p + q < |w|$, so $p < |w| - q$, implies, that $P \sqsubset vP^ky$, what is equivalent to $uv \sqsubset v(uv)^ky$, especially $uv \sqsubset vuv$, what gives us
      
      \begin{equation}
        uv = vu
      \end{equation}
      
      Basing on uv periods lemma and that $|u| = \textit{q mod p}$, enough to prove, that u period condition holds for $w[|w| - |y|, |w|$. As we know, that $y \sqsubset uv$, the proof is done.
    \end{proof}
    
    \section{}
    \section{}
    \section{}
    \section{}
      \begin{equation}
        \begin{split}
          \sum_{k=1}^n a_ir^{n-k} \textit{ mod q} &= (\dots((ra_1 + a_2)r + a_3)r \dots)r + a_n \textit{ mod q} \\ &= (\dots((ra_1 \textit{ mod q } + a_2)r \textit{ mod q } + a_3)r \textit{ mod q } \dots)r \textit{ mod q } + a_n 
\textit{ mod q }
        \end{split}
      \end{equation}
      
      n multiplications, n modulos, n - 1 additions gives us O(n) time. The second transformation is unnecessary. However, enables us to computing values less than $q^2$. If our $\Sigma\ll |S|$, then
      calculation all the $ra$\textit{ mod q } in preprocessing might be better than doing it ad hoc.
    \section{}
      Given $\phi_r(x)$ and $\phi_r(y)$:
      \begin{equation}\label{eq:phi}
        \begin{split}
          \phi_r(xy) &= \sum_{k=1}^{|xy|}S[k]r^{|xy| - k}\textit{ mod q } \\ &= (\sum_{k=1}^{|x|}S[k]r^{|xy| - k} + \sum_{k=|x| + 1}^{|xy|}S[k]r^{|xy| - k})\textit{ mod q } \\
          &= (r^|y|\sum_{k=1}^{|x|}x[k]r^{|x| - k}\textit{ mod q } + \sum_{k=|x| + 1}^{|xy|}S[k]r^{|xy| - k}\textit{ mod q })\textit{ mod q } \\ &= (r^{|y|}\phi_r(x) + \phi_r(y))\textit{ mod q }
        \end{split}
      \end{equation}
      
      Given $\phi_r(xy)$ and $\phi_r(y)$ and referencing \ref{eq:phi}:
      
      \begin{equation}
        \begin{split}
          \phi_r(xy) &\equiv r^{|y|}\phi_r(x) + \phi_r(y)\textit{ (mod q) } \\
          \phi_r(xy) - \phi_r(y) &\equiv r^{|y|}\phi(x)\textit{ (mod q) }
        \end{split}
      \end{equation}
      
      As we know, $q \in \mathbb{P}$ and $r \in \{1, 2, \dots, q - 1\}$. Hence, $gcd(r, q) = 1$. Considering, that we are in $\mathbb{Z}_p^*$, the inverse element of $r^{|y|}$, basing on Fermat's
      little theorem, is $r^{q - |y| - 1}$. Therefore
      
      \[\phi_r(x) \equiv r^{q - |y| - 1}(\phi_r(xy) - \phi_r(y))\textit{ (mod q) }\]
      
      As $\phi_r(x) < q$:
      
      \[\phi_r(x) = r^{q - |y| - 1}(\phi_r(xy) - \phi_r(y))\textit{ mod q }\]
      
      
\end{document}

