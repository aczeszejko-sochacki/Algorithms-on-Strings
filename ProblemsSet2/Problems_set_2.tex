\documentclass{scrartcl}
\usepackage[utf8]{inputenc}

\title{Algorithms on Strings}
\subtitle{Problems Set 2}
\author{Aleksander Czeszejko-Sochacki}
\date{October 2018}
\usepackage{amsmath}
\usepackage{amsthm}
\usepackage{amssymb}
\newtheorem{theorem}{Induction}

\begin{document}

\maketitle
\section{}
  \begin{proof}
    Since we know that $|t| \leq 2|p|$, each pair of the pattern occurrences has
    non-empty intersection.  Assume $q_1, q_2, \dots, q_n$ - all the occurrences of $p$ in $t$
    \begin{enumerate}
        \item $|t| < 2|p|$ \\ 
          Assume $q_1 = 0$. Consider two occurrences of $p$ in $t$ - $q_1, q_2$. 
          By the above fact there is a subword $w$ 
          having a border $|p|$. By the lemma from the Problem Set 1, we conclude, that $q_2$ - periods
          of $w$, so, especially, $q_1, q_2$ - periods of $p$. Considering all possible pairs of $q_i, q_j$ 
          gives us $q_1, \dots, q_n$ - periods of p. By the another lemma from the Problem Set 1,
          $gcd(q_i, q_j)$ - period of $p$. So $gcd(q_2, \dots, q_n) = d$ is also a period of p and hence
          \begin{equation}\label{period}
            p = u^kv
          \end{equation}
          where $|d| = u$ and
          \[q_i = \alpha d\]
          for all $i \in \{1, \dots, n\}$. Easy to prove, that $q_i = id$. Consider the first occurence such
          that $q_i = (i+k)d$, $k > 0$. Then there should be a mismatch at $id$, what stays in contradiction to matches at $q_{i-1},q_i$ and the form \ref{period}. \\
          For $q_1 \neq 0$ we need to subtract $q_1$ everywhere above.
        \item $|t| = 2|p|$ \\
          Trivial, there are only at most two occurrences of $p$ in $t$.
    \end{enumerate}
  \end{proof}

\section{}
  \begin{proof}
    Induction by $n$. \\
    We can write the thesis in an equivalent way:
    \[f_nf_{n+1} = c(f_{n+1}f_n)\]
    
    Observation:
    \begin{equation}
      c(ab) = ac(b) \textit{ if } |b| > 1
    \end{equation}
    
    \begin{theorem}[Base]
      \begin{equation}
        \begin{split}
          f_1f_2 &= c(f_2f_1)\\
          f2_1f_3 &= c(f_3f_2)
        \end{split}
      \end{equation}
    \end{theorem}
    \begin{proof}
      \begin{equation}
        \begin{split}
          ba &= c(ab)\\
          bab &= c(bba)
        \end{split}
      \end{equation}
    \end{proof}
    
    
    \begin{theorem}[Step]
      \begin{equation}
        f_nf_{n+1} = c(f_{n+1}f_n \implies f_{n+1}f_{n+1} = c(f_{n+2}f_{n+1})
      \end{equation}
    \end{theorem}
    \begin{proof}
      \begin{equation}
        \begin{split}
          f_{n+1}f_{n+2} &= f_{n+1}f_{n+1}f_n \\
          &= f_{n+1}f_nf_{n-1}f_n \\
          &= f_{n+2}f_{n-1}f_n && \text{induction}\\
          &= f_{n+2}c(f_nf_{n-1}) && \text{observation 1}\\
          &= f_{n+2}c(f_{n+1}) \\
          &= c(f_{n+2}f_{n+1})
        \end{split}
      \end{equation}
    \end{proof}
  \end{proof}
  
 \end{document}
